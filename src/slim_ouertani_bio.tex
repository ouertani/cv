% LaTeX file for resume
% This file uses the resume document class (res.cls)

\documentclass{res}

%\usepackage{helvetica} % uses helvetica postscript font (download helvetica.sty)
%\usepackage{newcent}   % uses new century schoolbook postscript font
\newsectionwidth{0pt}  % So the text is not indented under section headings
\usepackage{fancyhdr}  % use this package to get a 2 line header

\usepackage[colorlinks=false]{hyperref}
\usepackage{lipsum}
\usepackage{enumitem}


\renewcommand{\headrulewidth}{0pt} % suppress line drawn by default by fancyhdr
\setlength{\headheight}{24pt} % allow room for 2-line header
\setlength{\headsep}{24pt}  % space between header and text
\setlength{\headheight}{24pt} % allow room for 2-line header
\pagestyle{fancy}
\cfoot{}                                     % the foot is empty
\topmargin=-0.5in % start text higher on the page

\begin{document}
\thispagestyle{empty}
\begin{resume}

\section

Slim Ouertani is solution architect, InfoQ-Fr editor and SOA trainer. After his bachelor's degree in computer science, Slim spent more than 10 years between multinational companies as an architect, international agile coach and Head of IT development department. His main role was to build, design and collaborate in the development of enterprise solutions. Slim regularly contributes to InfoQ-FR community where he is an author of many articles and news. In addition to his SOA and Cloud certifications: Trainer, Architect \& Professional, Slim is certified Togaf 9, PMP, Spring, JAVA, ITIL, CMMi, Mongo...Slim enjoys open source contribution and invite you to follow him on twitter @ouertani.

\end{resume}
\end{document}
